\documentclass{article}
\usepackage[utf8]{inputenc}
\usepackage[spanish]{babel}
\usepackage{listings}
\usepackage{graphicx}
\graphicspath{ {images/} }
\usepackage{cite}

\begin{document}

\begin{titlepage}
    \begin{center}
        \vspace*{1cm}
            
        \Huge
        \textbf{Parcial-1 Calistenia}
            
        \vspace{0.5cm}
        \LARGE
        Desafío
            
        \vspace{1.5cm}
            
        \textbf{Maverick Sossa Tobon}
            
        \vfill
            
        \vspace{0.8cm}
            
        \Large
        Despartamento de Ingeniería Electrónica y Telecomunicaciones\\
        Universidad de Antioquia\\
        Medellín\\
        Marzo de 2021
            
    \end{center}
\end{titlepage}

\tableofcontents
\newpage
\section{Sección introductoria}\label{intro}
Hay un desafío: con una mano y sobre una hoja en blanco, colocar en forma de pirámide dos tarjetas de tal forma que se sostengan. Para superar el desafío se hace necesario crear un instructivo detallado que diriga a cualquier persona al resultado esperado. 

Partiendo de que ambas tarjetas estan juntas, sobre una superficie plana y con la hoja en blanco crubriendolas completamente, los pasos se darán a continuación.  

\section{Instructivo}
Importante: todo se hace con una sola mano.
\begin{enumerate}
    

\item Levantar la hoja y ponerla a un lado de las tarjetas. 

\item Poner, con una mano, ambas tarjetas sobre la hoja. Correr la hoja con las tarjetas un poco hacia sí mismo para comodidad. 

\item Levantar las tarjetas y juntarlas, una pegada a la otra.

\item Sujetar las tarjetas verticalmente de la siguiente forma:

\begin{itemize}
\item Los dedos pulgar y anular estarán en los bordes, en la zona de abajo de la tarjeta. 
\item El dedo índice en la parte superior en medio.
\end{itemize}
De modo que solo tres dedos estarán sosteniéndolas. 

\item Ponerlas sobe la hoja sosteniéndolas con la mano. Hacer un poco de presión sobe la hoja con las tarjetas para tener mayor firmeza.

\item Inclinar muy levemente las tarjetas hacia el lado contrario de la mano que las sujeta.  

\item Con los dedos pulgar y anular, desde la posicion en que estan, empezar a separar una tarjeta de la otra controladamente hasta tener una separación moderada, la tarjeta que se está separando no se debe arrastrar en la hoja, debe quedar en el aire. 



\item Soltar los dedos pulgar y anular. Solo debe quedar el índice en la parte de arriba.

Con el dedo indice hacer que las tarjetas queden parejas. 

\item Buscando la estabilidad, con el dedo índice controlar el movimiento de las tarjetas. Si se va hacia un lado moverlas levemente hacia el lado contrario hasta ver que se sostendrán por sí solas. 

\end{enumerate}
\section{Lo que aprendí}

Es claro, y necesario, que para llegar de un punto A a un punto B es muy importante pensar un plan y plasmarlo en un conjuto de pasos ordenados que marcan la ruta y la dirección a seguir. 

La cantidad de pasos a seguir depende principalmente de la ruta marcada, pero también del grado de precisión y complejidad adoptada, pues a mayor grado mayor cantidad de datos que detallar. Si tiene un grado bajo, que denota sencillez, los pasos serán menores y más entendibles, lo cual hace que sea más fácil seguir la ruta. 

Conseguir cierto grado de simplicidad hace más asequible llevar a cabo el plan y completarlo. Además, si se aprovecha la intuición de aquello que ejecuta el plan, hace que ciertos aspectos no se tengan que mencionar o detallar. Y es que detallar demasiado algo puede llevar fácilmente a la confusión. El hecho de que la comprensión y ejecución de las instrucciones esté subordinada a los previos conocimientos de quien los efectúa hace que sea necesario establecer estrategias para plantear tales pasos secuenciales lo más general posible.   


\end{document}
